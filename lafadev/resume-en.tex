%!TEX TS-program = xelatex
%!TEX encoding = UTF-8 Unicode
% Awesome CV LaTeX Template for CV/Resume
%
% This template has been downloaded from:
% https://github.com/posquit0/Awesome-CV
%
% Author:
% Claud D. Park <posquit0.bj@gmail.com>
% http://www.posquit0.com
%
% Template license:
% CC BY-SA 4.0 (https://creativecommons.org/licenses/by-sa/4.0/)
%


%-------------------------------------------------------------------------------
% CONFIGURATIONS
%-------------------------------------------------------------------------------
% A4 paper size by default, use 'letterpaper' for US letter
\documentclass[11pt, a4paper]{awesome-cv}

% Configure page margins with geometry
\geometry{left=1.4cm, top=.8cm, right=1.4cm, bottom=1.8cm, footskip=.5cm}

% fonts location
\fontdir[fonts/]

% Color for highlights
% Awesome Colors: awesome-emerald, awesome-skyblue, awesome-red, awesome-pink, awesome-orange
%                 awesome-nephritis, awesome-concrete, awesome-darknight
\colorlet{awesome}{awesome-darknight}
% Uncomment if you would like to specify your own color
% \definecolor{awesome}{HTML}{CA63A8}

% Colors for text
% Uncomment if you would like to specify your own color
% \definecolor{darktext}{HTML}{414141}
% \definecolor{text}{HTML}{333333}
% \definecolor{graytext}{HTML}{5D5D5D}
% \definecolor{lighttext}{HTML}{999999}
% \definecolor{sectiondivider}{HTML}{5D5D5D}

% Set false if you don't want to highlight section with awesome color
\setbool{acvSectionColorHighlight}{false}

% If you would like to change the social information separator from a pipe (|) to something else
\renewcommand{\acvHeaderSocialSep}{\quad\textbar\quad}


%-------------------------------------------------------------------------------
%	PERSONAL INFORMATION
%	Comment any of the lines below if they are not required
%-------------------------------------------------------------------------------
% Available options: circle|rectangle,edge/noedge,left/right
% \photo[rectangle,edge,right]{./examples/profile}
\name{Lucas}{Flores}
\position{Full Stack Developer{\enskip\cdotp\enskip}}
\address{Rio de Janeiro, Brazil}

%-------------------------------------------------------------------------------
\begin{document}
\newline
%\newline
% Print the header with above personal information
% Give optional argument to change alignment(C: center, L: left, R: right)
\makecvheader[L]
%\begin{picture}(0,0)
%    \put(440,-30){\includegraphics[width=7em]{seloCPA.jpg}}
%\end{picture}
%\linebreak
\begin{wrapfigure}{l}{0.1\textwidth}
  \begin{center}
    \href{https://wa.me/message/7QZXUVB4LDJMH1}{\includegraphics[width=0.10\textwidth]{QRcode3.png}}
  \end{center}
\end{wrapfigure}
\\
\href{tel:21966637783}{\faPhone\acvHeaderIconSep\ (21) 96663-7783}
\hspace{0.5cm}
\href{mailto:lafa.dev@protonmail.com}{\faEnvelope\acvHeaderIconSep\ lafa.dev@protonmail.com}
\hspace{0.5cm}
\\
\\
\href{https://www.linkedin.com/in/lafa}{\faLinkedinSquare\acvHeaderIconSep\@ Lucas Flores}
\hspace{1.1cm}
\href{https://github.com/LafaDev}{\faGithubSquare\acvHeaderIconSep\ /LafaDev}
\hspace{2cm}
\\

% Print the footer with 3 arguments(<left>, <center>, <right>)
% Leave any of these blank if they are not needed

\makecvfooter
  {}
  {\LaTeX~~~Résumé~~~·~~~Lucas Flores}
  {}


%-------------------------------------------------------------------------------
%	CV/RESUME CONTENT
%	Each section is imported separately, open each file in turn to modify content
%-------------------------------------------------------------------------------
%-------------------------------------------------------------------------------
%	SECTION TITLE
%-------------------------------------------------------------------------------
\cvsection{About Me}


%-------------------------------------------------------------------------------
%	CONTENT
%-------------------------------------------------------------------------------
\begin{cvparagraph}

%---------------------------------------------------------

I'm a full-tack developer. At the moment, i study at Trybe, a programming school where
i got the experience of working with different people on many projects.
I love reading, gaming and i'm always learning new things.
\end{cvparagraph}

\cvsection{Habilidades}
\cvsubsection{Programação}
\begin{cvhonors}
  \cvhonor
    {Node.js, MySQL, Docker}
    {JWT, Bash, Heroku}
    {}
    {Ling:}
  \cvhonor
    {Arquitetura MSC, Arquitetura Rest e Restful}
    {}
    {}
    {Web:}
\end{cvhonors}

%\cvsubsection{Database Management System (DBMS)}
%\begin{cvhonors}
  %\cvhonor
    %{MySQL, Oracle}
    %{}
    %{}
    %{Proficient}
  %\cvhonor
    %{SQLite}
    %{}
    %{}
    %{Familiar}
%\end{cvhonors}

\cvsubsection{Frameworks, Bibliotecas \& Arquiteturas}
\begin{cvhonors}
  \cvhonor
    {Sequelize}
    {React, Redux}
    {}
    {Software:}
 % \cvhonor
    %{}
    %{}
    %{}
    %{Familiar}
\end{cvhonors}
\cvsubsection{Ferramentas de Desenvolvimento}
\begin{cvhonors}
  \cvhonor
    {Visual Studio Code, Vim, NeoVim, Terminal (Linux)}
    {}
    {}
    {Editores:}
  %\cvhonor
    %{Emacs, Notepad++}
    %{}
    %{}
    %{Familiar}
\end{cvhonors}

%\cvsubsection{Game Engines}
%\begin{cvhonors}
  %\cvhonor
    %{Unity 3D, Unity 2D}
    %{}
    %{}
    %{Proficient}
  %\cvhonor
    %{Game Maker Studio, Unreal Engine}
    %{}
    %{}
    %{Familiar}
%\end{cvhonors}

\cvsubsection{Sistema De Versionamento de Codigo}
\begin{cvhonors}
  \cvhonor
    {Git (GitHub)}
    {}
    {}
    {Software:}
\end{cvhonors}

\cvsubsection{Outras Linguagens}
\begin{cvhonors}
  \cvhonor
    {Tex (LaTex)}
    {}
    {}
    {Ling:}
\end{cvhonors}

\cvsubsection{Sistemas}
\begin{cvhonors}
  \cvhonor
    {Gentoo Linux, Ubuntu, Arch Linux}
    {Debian, Manjaro}
    {}
    {Linux:}
  \cvhonor
    {Windows}
    {}
    {}
    {Outros:}
\end{cvhonors}

\cvsubsection{Idiomas}
\begin{cvhonors}
  \cvhonor
    {Avançado}
    {}
    {}
    {Inglês:}
\end{cvhonors}

\vspace{.5cm}

\cvsection{Certifications}
\begin{cvhonors}
\cvsubsection{ANBIMA}
  \cvhonor
    {Professional Certification from Brazilian Association of Financial and Capital Market Institutions}
    {(CPA-20)}
    {2020 - 2023}
    {}
\cvsubsection{Trybe}
  \cvhonor
    {Web Development Basics}
    {}
    {}
    {}
  \cvhonor
    {Front-End Development}
    {}
    {}
    {}
\end{cvhonors}

\vspace{.5cm}
%-------------------------------------------------------------------------------
%	SECTION TITLE
%-------------------------------------------------------------------------------
\cvsection{Formação}


%-------------------------------------------------------------------------------
%	CONTENT
%-------------------------------------------------------------------------------
\begin{cventries}

%---------------------------------------------------------
\cventry
    {Desenvolvimento Web Full Stack} % Degree
    {Trybe} % Institution
    {Escola De Programação} % Location
    {Setembro 2021 - Presente} % Date(s)
    {
      \begin{cvitems} % Description(s) bullet points
        \item {O programa conta com mais de 1.500 horas de aulas.}
      \item {Introdução ao desenvolvimento de software, Front End, Back End e engenharia de software.}
      \item {Desenvolvimento de Soft Skills, Metodologias ágeis e habilidades comportamentais.}
        %\item {Metodologias ágeis.}
      \end{cvitems}
    }
   
   \cventry
    {Bacharelado em Ciências Contábeis} % Degree
    {Universidade Federal Rural do Rio de Janeiro} % Institution
    {Seropédica} % Location
    {2018 - 2020 (Incompleto)} % Date(s)
    {
      \begin{cvitems} % Description(s) bullet points
        \item{Conhecimentos sobre mercado financeiro}
      \end{cvitems}
    } 

%---------------------------------------------------------
\end{cventries}



%-------------------------------------------------------------------------------
\end{document}
